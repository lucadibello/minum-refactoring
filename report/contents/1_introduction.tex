This assignment requires the use of the knowledge acquired during the course, in order to refactor an existing open-source project, aiming to improve its design. The behavior of the project should remain unchanged, as well as its input and output interfaces.

The refactoring should target at least 1000 lines of code, and the changes should be documented in a report, and pushed to a separate branch in the project's repository, allowing an easy comparison between the original and the new, improved version.

To find valuable candidates for this assignment, the GitHub search feature was used, filtering the results by language, Java, and the number of stars, between 100 and 1000. The search results were sorted by last update date, in descending order in order to find active projects.

To conclude, each selected project size was analyzed with the web application \href{https://codetabs.com/count-loc/count-loc-online.html}{Count LOC}, in order to count the total lines of code (later referred as \emph{LOC}) that would be affected by the refactor (considering only source code, excluding tests or other utilities).

\subsection{Project selection}

As cited before, in order to accomplish the personal objective set at the beginning of this assignment, the search targeted project with a high number of stars, between 100 and 1000, and written in Java language in order to leverage the knowledge acquired during the course. The search results were sorted by the last update date, in descending order, to find active projects. After an in-depth analysis of results, these are the possible candidates selected for further investigation:

\begin{itemize}
	\item \href{https://github.com/flanglet/kanzi}{fanglet/kanzi}: Kanzi is a modern, modular and efficient lossless data compressor implemented entirely in java. It uses state-of-the art entropy coders and multi-threading in order to efficiently utilize multi-core CPUs. The design of the library is modular, allowing to select at runtime the best entropy coder for the data to compress. The project has 109 stars, 18 forks, no open issues, and approximately 20,000 LOC. Kanzi was initially selected for this assignment, but later discarded due to its size.
	\item \href{https://github.com/jinput/jinput}{jinput/jinput}: JInput is a Java library designed for accessing input devices such as game controllers, joysticks, and other peripherals. It provides a platform-independent API to facilitate the integration of various input devices into Java applications. The project has 150 stars, 79 forks, 29 open issues, and 10,000 lines of code (considering only core functionality, excluding tests). JInput was initially selected for this assignment, but unfortunately, after inspecting the codebase it was discarded as the project presented too many platform-specific implementations spanning multiple modules and a complete lack of documentation and tests. This would make the refactoring process too complex and time-consuming.
	\item \href{https://github.com/byronka/minum}{byronka/minum}: Minum is a minimalistic web framework written in Java, built from scratch using few dependencies. The project provides essential components for web application development, including a web server and an in-memory database with disk persistence. This project was particularly interesting as it emphasizes simplicity and minimalism, which is a good starting point for refactoring as most of the code is written from scratch without complex dependencies. Minum has 611 stars, 38 forks, no open issues, and approximately 5'500 LOC (excluding tests and comments). After a thorough analysis, this project was selected for refactoring as it presents a clear and concise codebase divided into several packages, which will allow for a more focused refactor. Furthermore, it presents a complete JavaDoc documentation, which will be useful to understand the project's design.
\end{itemize}

\noindent \textit{Note}: The data presented above was collected on the \displaydate{collectiondate}, and may have changed since then. Check the project's repository for the most recent information.

\subsection{High-level overview of the project structure}
\label{sec:project_structure}

The project presents a single \textit{Maven} module which contains the core functionality of the framework. The module contains 10 packages, each with a specific purpose. The following list provides a high-level overview of the purpose of each package:

\vspace{1em}
\noindent
\begin{minipage}{0.5\textwidth}
	\begin{itemize}
		\item \texttt{database}: In-memory database with disk persistence.
		\item \texttt{htmlparsing}: HTML parsing utilities.
		\item \texttt{logging}: Logging utilities.
		\item \texttt{queue}: Queue of tasks to be executed asynchronously.
		\item \texttt{security}: \href{https://github.com/fail2ban/fail2ban}{Fail2Ban}-like security mechanism and threat detection via log analysis.
	\end{itemize}
\end{minipage}
\begin{minipage}{0.5\textwidth}
	\begin{itemize}
		\item \texttt{state}: Singleton context to store global state.
		\item \texttt{templating}: Template engine to inject dynamic content into HTML pages.
		\item \texttt{testing}: Minimalistic testing framework for unit and integration tests.
		\item \texttt{utils}: General utilities.
		\item \texttt{web}: Web server and request handling.
	\end{itemize}
\end{minipage}
\vspace{1em}

\noindent From a first glance, the project seems well-structured, with clear separation of concerns between packages. As cited before, the codebase is well-documented, with JavaDoc comments present in most classes and methods, providing a good starting point to understand the project's design. In \autoref{sec:project_health_analysis}, the project design will be analyzed in more detail, focusing on the use of design patterns and potential code smells.

\subsection{Additional tools and resources}

In order to perform a comprehensive refactor of the project, the \href{https://www.sonarsource.com/}{SonarQube} static code analysis tool will be used to identify potential code smells, bugs, and vulnerabilities, while also providing insights into the overall code quality. Additionally, in order to have a more in-depth understanding of the current design of the project, the \textit{Pattern4j} tool will be used to detect the use of design patterns in the codebase, an essential aspect of the refactoring process.

The results of both tools provide valuable information to guide the refactor, highlighting areas of improvement and potential refactoring opportunities, and, by combining both outputs, will be possible to have a more comprehensive view of the project's design and code quality.

\subsection{Building the project}

The Minum project uses Maven as the build system, and in order to simplify the configuration process, the creator of the library created a Maven Wrapper script (named \texttt{mvnw}) which is a self-container script that allows to automatically download the necessary Maven version to build the project. This ensures that the build process is reproducible across different environments. This script is located in the root of the project, and can be used to build the project by running the following command:

\begin{center}
	\begin{minipage}{0.5\textwidth}
		\begin{lstlisting}[language=bash, caption={Building the project using the Maven Wrapper script}]
    ./mvnw clean install
  \end{lstlisting}
	\end{minipage}
\end{center}

\noindent \textit{Note}: If the current environment already has Maven installed, the project can be built using the \texttt{mvn} command instead of the wrapper script.
\vspace{1em}

\noindent The build artifacts can be found in the \texttt{target} directory. The bytecode generated by the build process will be used by the \textit{Pattern4j} tool to analyze the project's design.
