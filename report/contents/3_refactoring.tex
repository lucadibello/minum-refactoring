In the following subsections, each step of the planned refactoring presented in \autoref{sec:refactoring_plan} will be detailed, providing insights into the previous state of the project, and the results achieved though the refactoring.

\subsection{\texttt{logging} package refactoring}

Within this package, the \texttt{Logger} class supports four logging levels: \texttt{DEBUG}, \texttt{TRACE}, \texttt{AUDIT}, and \texttt{ASYNC\_ERROR}, which are defined in the \texttt{LoggingLevel} \texttt{Enum}. To enhance flexibility, the library maintains the current state of these logging levels in a \texttt{HashMap}, where each entry is represented as a record \texttt{<LoggingLevel, Boolean>}. In this mapping, the key corresponds to a specific logging level as an \texttt{Enum}, and the value is a boolean indicating whether that level is enabled.

The library also includes a test example that demonstrates how to extend the built-in logger by defining a \texttt{DescendantLogger} class, which extends the \texttt{Logging} class in order to support a new set of logging levels encoded as the \texttt{CustomLoggingLevel} \texttt{Enum}. However, since there is no inheritance between \texttt{CustomLoggingLevel} and \texttt{LoggingLevel}, the library requires creating a separate \texttt{HashMap} to store the new logging levels. This approach results in code duplication, as the \texttt{DescendantLogger} class must reimplement all methods for managing logging levels, such as enabling/disabling levels, checking if a level is enabled, and logging messages at specific levels. This violates the DRY (\href{https://en.wikipedia.org/wiki/Don%27t_repeat_yourself}{Don't Repeat Yourself}) principle and increases the overall complexity of the codebase.

To address these design issues and create a more flexible and maintainable solution, a new abstract class, \texttt{BaseLogger}, was created. This class encapsulates the necessary data structures to manage any set of logging levels defined as \texttt{Enum} objects implementing the \texttt{ILoggingLevel} interface. This allows to store logging levels in a single \texttt{HashMap} regardless of their type, thus reducing code duplication and improving maintainability. Furthermore, a new generic interface, \texttt{ILogger<ILoggingLevel>}, was introduced. This interface includes a method \texttt{log(String, ILoggingLevel)}, which is implemented by the \texttt{BaseLogger} class to log messages at a specific logging level. This improved design is illustrated in \autoref{fig:refactored-logging}.

% Logging package original
\begin{minipage}{0.4\linewidth}
	\begin{figure}[H]
		\begin{center}
			\includegraphics[width=0.95\textwidth]{figures/logging_package/original.png}
			\caption{Original \texttt{logging} package}
			\label{fig:original-logging}
		\end{center}
	\end{figure}
\end{minipage}
\begin{minipage}{0.6\linewidth}
	\begin{figure}[H]
		\begin{center}
			\includegraphics[width=0.95\textwidth]{figures/logging_package/refactored.png}
			\caption{Refactored \texttt{logging} package}
			\label{fig:refactored-logging}
		\end{center}
	\end{figure}
\end{minipage}

\noindent By adopting this approach, developers can define custom logger classes by simply creating new logging levels in an \texttt{Enum} that extends the \texttt{ILoggingLevel} interface. Each custom logger will only need to implement specialized log methods for each logging level. This allows to create custom logger classes with minimal code duplication, ensuring adherence to the DRY principle.

As the \texttt{Logger} class was extensively used inside the core library, the refactoring process required updating all references to the \texttt{Logger} class to use the new \texttt{CanonicalLogger} class, which defines the default logging levels present in the original implementation to ensure backward compatibility. This was a long and tedious process, but it was necessary to ensure that the library maintained the same behavior as before.

Furthermore, also the proposed example of extension of the \texttt{Logger} class was refactored. Now, the \texttt{CanonicalLogger} class can be extended to support custom logging levels by simply defining a new \texttt{Enum} that implements the \texttt{ILoggingLevel} interface. This architecture allows an easy extension and customization of logging levels, as it is only necessary to define a new \texttt{Enum} that extends the \texttt{ILoggingLevel} interface and implement the required methods.

In the new test case, \texttt{DescendantCanonicalLogger} (previously \texttt{DescendantLogger}) implements an additional logging level \texttt{REQUEST} while still supporting the logging levels of the \texttt{CanonicalLogger} class. This demonstrates how the new architecture simplifies the process of extending the \texttt{Logger} class to support custom logging levels.

\subsection{Exception classes refactoring}

All exception classes are now in a single \texttt{exception} package in order to tidy up the codebase. The original design had exception classes scattered throughout the codebase, which made it difficult to locate and manage them.

\subsection{\texttt{queue} package refactoring}

The library defines a class \texttt{ActionQueue} that represents a queue of actions to be executed asynchronously by a thread. It receives a list of actions and executes them in the order they were added. For the development of the library, it was necessary to create a new class, \texttt{LoggingActionQueue}, which executes actions in a slight different way than \texttt{ActionQueue}. With the current architecture, \texttt{LoggingActionQueue} is almost an exact copy of \texttt{ActionQueue}. Refer to \autoref{fig:original-queue} for UML diagram of the original design.

To solve this issue, a new abstract class, \texttt{BaseActionQueue}, was created. This class encapsulates the common logic between \texttt{ActionQueue} and \texttt{LoggingActionQueue}, allowing to reduce code duplication and improve maintainability. After further analysis, was possible to remove completely the \texttt{LoggingActionQueue} class, as it was no longer necessary. The new design is illustrated in \autoref{fig:refactored-queue}.

% Logging package original
\begin{minipage}{0.5\linewidth}
	\begin{figure}[H]
		\begin{center}
			\includegraphics[width=0.95\textwidth]{figures/queue_package/original.png}
			\caption{Original \texttt{queue} package}
			\label{fig:original-queue}
		\end{center}
	\end{figure}
\end{minipage}
\begin{minipage}{0.5\linewidth}
	\begin{figure}[H]
		\begin{center}
			\includegraphics[width=0.95\textwidth]{figures/queue_package/refactored.png}
			\caption{Refactored \texttt{queue} package}
			\label{fig:refactored-queue}
		\end{center}
	\end{figure}
\end{minipage}
