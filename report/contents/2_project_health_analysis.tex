The project will be analyzed using both \href{https://www.sonarsource.com/}{SonarQube} and \textit{Pattern4j} static code analysis tools, and also by custom-made shell scripts that collect interesting metrics about the codebase.

In the following section, the results of the analysis will be presented, highlighting potential code smells, bugs, vulnerabilities, and design patterns detected in the project.

As the configuration and usage of both tools is out of the scope of this report, the following sections will focus on the results of the analysis, providing insights into the current state of the project, and guiding the refactoring process. Refer to the respective documentation for more information on how to configure and use both tools.

\subsection{Static Analysis Tools}

\subsubsection{Large class detection}

In order to detect large classes, a custom bash script was developed to count the number of lines of code of each class in the project. The script uses the \texttt{cloc} tool to count the number of lines of code of each file in the project, and then orders the results by the number of lines of code in order to detect the largest classes. The following are the top 4 largest classes in the project (excluding test classes and comments):

\begin{enumerate}
	\item \texttt{com.renomad.minum.web.WebFramework} - 415 LOC
	\item \texttt{com.renomad.minum.htmlparsing.HtmlParser} - 372 LOC
	\item \texttt{com.renomad.minum.database.} - 219 LOC
	\item \texttt{com.renomad.minum.web.Response} - 214 LOC
\end{enumerate}

\noindent The average number of lines of code is 55 LOC, which is considered acceptable. Due to the size of the project, the refactoring process will focus on the \texttt{Response} class, as it is the largest class in the project and is extensively used in the core package of the library, providing a good starting point.

\subsubsection{Code duplication and inheritance misuse}

By manually analysing the codebase, was found that inside the \texttt{logging} and \texttt{queue} package  there are multiple classes that share the same methods, which could be extracted into a common superclass in order to provide a more modular and maintainable codebase. As these two packages comprehend a total of 8 classes, this could be a good starting point for the refactoring process.

\subsubsection{SonarQube analysis}

After running the SonarQube analysis on the entire Minium codebase (including tests in order to get the test coverage metric), were detected a total of 588 code smells, 26 security hotspots and 34 possible bugs. In the following paragraphs the results will be briefly analyzed in order to plan the refactoring process. \autoref{tab:sonarqube_severity_summary} provides a summary of the found code smells, categorized by severity.

\begin{table}[H]
	\centering
	\caption{SonarQube Severity Issues Summary}
	\label{tab:sonarqube_severity_summary}
	\begin{tabular}{|c|p{14cm}|}
		\hline
		\textbf{Severity Type} & \textbf{Issues}                                                \\ \hline
		\textbf{Critical}      &
		\begin{tabular}[t]{@{}l@{}}
			design (90), suspicious (10), brain-overload (6) convention (1), multi-threading (1), \\
			pitfall (1)                                                                           \\
		\end{tabular}   \\ \hline
		\textbf{Major}         &
		\begin{tabular}[t]{@{}l@{}}
			cert (40), html5 (20), obsolete (19) bad-practice (17), owasp-a3 (17), cwe (16), \\
			error-handling (15), pitfall (8), suspicious (7) accessibility (5), unused (4),  \\
			wcag2-a (4) confusing (2), design (2), brain-overload (1)                        \\
		\end{tabular}        \\ \hline
		\textbf{Minor}         &
		\begin{tabular}[t]{@{}l@{}}
			convention (374), cwe (7), java8 (5), brain-overload (4), pitfall (4), performance (2), \\
			regex (2), unused (2), bad-practice (1), clumsy (1), suspicious (1)                     \\
		\end{tabular} \\ \hline
	\end{tabular}
\end{table}

\noindent As shown in Table \ref{tab:sonarqube_severity_summary}, the most common code smells in the project are related to Java conventions, design issues, CERT secure coding standards. During the refactoring of the project, these issues will be taken into account in order to improve the overall quality of the codebase.

\subsubsection{Pattern4j analysis}

Unfortunately, the pattern4j tool was not able to analyze the entire project codebase due to the use of Java 21 features in the project. By running the tool using the custom \texttt{run-pattern4j-headless.sh} script, the following error was raised:

\begin{verbatim}
java.lang.IllegalArgumentException: Unsupported class file major version 65
\end{verbatim}

\noindent For this reason, the design pattern usage anaylsis will be skipped. Fortunately, this step is not crucial for the refactoring of the project as it would have only provided additional insights into the design of the project rather than pinpointing specific issues.

\subsubsection{Codebase structural problems}

As cited in \autoref{sec:project_structure}, the project codebase is divided into 10 packages, each with a specific purpose. The overall structure of the codebase is showcase a very thorough organization, with each package containing a set of classes and interfaces related to a specific aspect of the library. However, by inspecting in more detail the structure of single packages four main issues were identified:

\begin{enumerate}
	\item The framework defines many custom exceptions in order to handle specific errors. These exceptions are scattered across the codebase and are not properly organized. To address this issue, a new package should be created to contain all the custom exceptions, and the classes should be moved there.
	\item Inside the \texttt{logging} package there are classes which are specific for testing purposes, which should be moved to the test package.
	\item Inside the \texttt{util} package there are many self-contained utility classes that do not properly follow the single responsibility principle. Most of these classes contain multiple methods that are not related to each other, and should be split into multiple classes in order to improve the overall design of the project.
\end{enumerate}

\subsection{Refactoring Plan}
\label{sec:refactoring_plan}

As outlined in the previous sections, the Minium project presents several code smells and design issues that need to be addressed in order to improve the overall quality of the codebase. The refactoring will aim to solve the issues related to the \texttt{logging} package by providing an hierarchy of classes that will allow to better manage the logging system of the library, and allow to easily extend it in the future. The testing classes present in the \texttt{logging} package will be moved to the relative testing package.

Furthermore, the custom exceptions will be moved to a new package in order to better organize the codebase and, to conclude, the \texttt{Response} object will be split into multiple classes in order to reduce its complexity.

