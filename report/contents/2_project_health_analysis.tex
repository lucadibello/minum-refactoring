The project will be analyzed using both \href{https://www.sonarsource.com/}{SonarQube} and \textit{Pattern4j} tools. In the following section, the results of the analysis will be presented, highlighting potential code smells, bugs, vulnerabilities, and design patterns detected in the project.

As the configuration and usage of both tools is out of the scope of this report, the following sections will focus on the results of the analysis, providing insights into the current state of the project, and guiding the refactoring process. Refer to the respective documentation for more information on how to configure and use both tools.

\subsection{SonarQube analysis}

After running the SonarQube analysis on the \texttt{CoreAPI} module, were detected a total of 26 code smells and 3 security hotspots. The follwing paragraphs will provide an overview of the most relevant code smells detected in the project, and the actions to be taken to address them.

\paragraph*{Code smells} Out of the 26 code smells detected in the project, 11 of them are categorized as critical, 4 as major, and 11 as minor. The following table provides a summary of the found code smells, categorized by severity.

\begin{table}[H]
	\centering
	\label{tab:sonarqube_severity_summary}
	\begin{tabular}{|c|c|}
		\hline
		\textbf{Severity Type} & \textbf{Issues (Count)} \\ \hline
		\textbf{Major}         &
		\begin{tabular}[t]{@{}l@{}}
			Bad Practice (3), Performance (3), Pitfall (3), \\
			CERT (2), Error Handling (2), Confusing (1),    \\
			CWE (1), Design (1), Suspicious (1)
		\end{tabular}  \\ \hline
		\textbf{Critical}      &
		Brain Overload (2), CERT (1), Suspicious (1)     \\ \hline
		\textbf{Minor}         &
		Convention (9), Brain Overload (1), Clumsy (1)   \\ \hline
	\end{tabular}
	\caption{SonarQube Severity Issues Summary}
\end{table}

\noindent As shown in Table \ref{tab:sonarqube_severity_summary}, the most common code smells in the project are related to bad practices, performance, and error handling. As the total number of code smells is relatively small, the refactoring will take into account all the detected issues, with a special focus on the critical and major ones.

\paragraph*{Security hotspots} The SonarQube analysis also detected 4 security hotspots in the project. Three of them are categorized as \textit{Insecure configuration}, and one as \textit{Log Injection}. The three insecure configuration hotspots are all coming from a common problematic pattern used in the \texttt{ControllerEnvironment} class, where, in case of an exception, the stack trace is printed to the standard output. This behavior can lead to information disclosure, and should be addressed as soon as possible.

The log injection issue on the other hand is related to the use of the \texttt{Logger} class without proper sanitization of the logger name, which can lead to log injection attacks.

Both security hostspots will be addressed in the refactoring process in order to enhance the security of the project.

\subsection{Pattern4j analysis}

After running the Pattern4j tool on the bytecode of the \texttt{CoreAPI} module, a total of 4 design patterns were detected. These include the \textit{Singleton}, \textit{Adapter}, \textit{Observer}, and \textit{Template Method} patterns. Each pattern contributes to the structure and functionality of the module by addressing specific design challenges, such as global instance management, interface adaptation, event-based notification, and algorithm reuse. The following table provides a categorized summary of the detected patterns and their respective roles.

\begin{table}[h!]
	\centering
	\label{tab:pattern_summary}
	\begin{tabular}{|l|l|}
		\hline
		\textbf{Design Pattern} & \textbf{Purpose}                                                                           \\ \hline
		Singleton               & Ensures a single instance of \texttt{ControllerEnvironment}.                               \\ \hline
		Adapter                 & Adapts \texttt{Event} and \texttt{EventQueue} for \texttt{AbstractController}.             \\ \hline
		Observer                & Implements event-based notification for controller state changes.                          \\ \hline
		Template Method         & Defines reusable algorithms in \texttt{AbstractComponent} and \texttt{AbstractController}. \\ \hline
	\end{tabular}
	\caption{Detected Design Patterns in \texttt{CoreAPI}}
\end{table}

The presence of these design patterns indicates a structured and well-organized design in the \texttt{CoreAPI} module, providing a foundation for extensibility and maintainability. In the refactoring process, the existing patterns will be preserved and potentially enhanced to improve code clarity and flexibility. Furthermore, new patterns may be introduced to address emerging design challenges and promote code reuse.

\subsection{Project structure}

As cited in \autoref{sec:project_structure}, the \texttt{CoreAPI} module presents a monolithic design, with several classes and interfaces presented in a single package. This design choice may lead to a lack of modularity and separation of concerns, making the codebase harder to maintain and extend. The refactoring process will focus on addressing this issue by splitting the module into smaller, more manageable components, each responsible for a specific aspect of the library.

